%% Conjuntos
\newcommand{\partes}{\ensuremath{\mathcal{P}}} % Conjunto de partes

%% Operacione enteros
\newcommand{\divcom}[2]{DivCom(\{#1,#2\})}  % Conjutno de divisores comunes entre #1 y #2
\newcommand{\divcomp}[2]{DivCom_{+}(\{#1,#2\})} % Conjunto de divisores comunes positivos #1 y #2
\newcommand{\divp}[1]{Div_{+}(#1)} 	% Divisores positivos de #1
\newcommand{\modulo}[3]{#1 \equiv #2 ~(\texttt{mod } #3)} % m congruente a n modulo p
\newcommand{\moduloI}[3]{#1 \equiv #2 ~(#3)} % m congruente a n modulo p (short version)
\newcommand{\nmodulo}[3]{#1 \not\equiv #2 ~(\texttt{mod } #3)} 	% m no congruente a n modulo p

%%Formas de complejos
\newcommand{\trigform}[3]{\ensuremath{#1=#2(\cos(#3) + i\sen(#3))}} % Forma trigonometrica 
\newcommand{\eulerform}[2]{\ensuremath{#1e^{#2 i}}} 				% Forma euler
\newcommand{\cuerpo}[2]{#1\left[#2\right]}	 						% Definicion de un cuerpo

%%% Simbolos
\newcommand{\comp}[4]{\ensuremath{#1\circ #2 : #3\rightarrow #4}} 	% Composición de funciones
\newcommand{\rel}{\ensuremath{\mathcal{R}}} 						% Símobolo de relacion

%%% Tipos
\newcommand{\func}[3]{\ensuremath{#1 : #2\rightarrow #3}}  	% f: A -> B
\newcommand{\complejos}{\ensuremath{\mathbb{C}}} 			% Complejos
\newcommand{\ent}{\ensuremath{\mathbb{Z}}} 					% Enteros
\newcommand{\nat}{\ensuremath{\mathbb{N}}} 					% Naturales
\newcommand{\racionales}{\ensuremath{\mathbb{Q}}} 			% Racionales
\newcommand{\reales}{\ensuremath{\mathbb{R}}} 				% Reales
