\paragraph{Apéndices}
\section{Otras funciones primitivas recursivas:}\label{appendix::funcpr}
\begin{multicols}{2}
\subsection{Division}
$y$ div $y = \min\limits_{t \leq x} x*(t+1) > y $

\subsection{Divisor}
$x | y = (\exists t)_{\leq y}~x*t = y $

\subsection{$i$-ésimo primo}
$p_0 = 2$

$p_{t+1} = \min\limits_{k \leq p_t!+1}(k > p_t \land \text{esPrimo}(p_k)$
\end{multicols}
\section{Otras Macros} \label{appendix::otrasMacros}
\setlength\columnsep{10pt}

\subsection{$V \leftarrow 0$}\label{appendix::otrasMAcros::resetVariable}
\begin{multicols}{2}
	\begin{center}
	\begin{tabular}{ll}
	$[A]$ & $\sdecr{V}$ \\
	& $\sif{V}{A}$
\end{tabular}

	\end{center}
	
	\columnbreak
La macro resta uno a $V$ hasta que sea cero.
\end{multicols}

\subsection{$V \leftarrow k$}\label{appendix::otrasMAcros::asignarConstante}
\begin{multicols}{2}
	\begin{center}
	\begin{tabular}{l}
	$V \leftarrow 0$ \\
	$\sincr{V}$ \\
	$\sincr{V}$ \\
	$\vdots$ \\
	$\sincr{V}$ \\
\end{tabular}
	\end{center}
	
	\columnbreak
	La macro asigna cero a $V$ y luego la incrementa en uno $k$ veces.
\end{multicols}

\subsection{$x = 0$}
\begin{multicols}{2}
	\begin{center}
		\begin{tabular}{l}
			\sif{X}{E}\\
			\sincr{Y} \\
		\end{tabular}
	\end{center}

\columnbreak
Este programa recibe $X$ como parámetro. Devuelve $1$ si $X$ es igual a cero y devuelve $0$ si no.
\end{multicols}
\subsection{$V \leftarrow f(V_1,...,V_n)$}\label{appendix::otrasMAcros::asignacionFunc}

Dada $f$ una función parcialmente computable queremos asignar a $V$ el resultado de pasarle como parámetro las variables $V_1,\dots,V_n$.
	
Como $f$ es parcial computable, sabemos que existe un programa $P$ tal que $f(x_1,\dots,x_n) = \Psi_P^{(m)}(x_1,\dots,x_n)$. Asumamos que $P$ usa las variables $X_1,\dots, X_m$, $Z_1,\dots,Z_k$ e $Y$, además de las etiquetas $E, A_1,\dots,A_l$. 

Además, supongamos que para cada instrucción de la forma $\sif{V}{A_i}$, la etiqueta referenciada se encuentra dentro del programa $P$, por lo que la única etiqueta de \textit{salida} es $E$.

Entonces, si tenemos un programa $P_0$ en el que queremos almacenar el resultado de una función, solo hace falta renombrar las variables y etiquetas usadas por $P$ a variables y etiquetas que no hayan sido usadas por $P_0$. 

Sea $m$ un número lo suficientemente grande como para que las variables y etiquetas que remplazamos no estén usadas por $P_0$ entonces, podemos hacer los siguientes remplazos: $Y \to Z_m$, $X_i \to Z_{m + i + 1}$, $Z_i = Z_{m + n + 1 + i}$ y $A_i \to A_{m + i}$.

\begin{multicols}{2}
\begin{center}
\begin{tabular}{ll}
	&$Z_m \leftarrow 0$ \\
	&$Z_{m+1} \leftarrow V_1$ \\
	&$Z_{m+2} \leftarrow V_2$ \\
	&$\vdots$ \\
	&$Z_{m+n} \leftarrow V_n$ \\
	&$Z_{m+n+1} \leftarrow 0$ \\
	&$\vdots$ \\
	&$Z_{m+n+k} \leftarrow 0$ \\
	&$P$ \\
	$[E_m]$ & $W\leftarrow Z_m$ \\ 
\end{tabular}
\end{center}

\columnbreak
Primero \textit{inicializamos} las variables de $P$. $Z_{m}$ es donde el programa $P$ (con los remplazos) va a guardar su resultado y $Z_{m+1}$ a $Z_{m+n}$ son las variables que remplazaron a los parámetros y $Z_{m+n+1}$ a $Z_{m+n+k}$ las variables locales. Además, de los remplazos de las etiquetas $A_i$ dentro de $P$, remplazamos $E$ por $E_m$.

De esta forma, cuando $P$ termina tanto porque se acabó la lista de instrucciones o porque hizo un salto a $E_m$ asignamos el valor de $Z_m$ a $W$.
\end{multicols}
Si quisieramos ejecutar $P$ dentro de un loop, cada vez que termine va a dejar los valores de la última ejecución en las variables que use. Por esta razón es necesario realizar la \textit{inicialización}.


\subsection{IF ${p(\xDots{V}{n})}$ GOTO $L$}\label{appendix::otrasMAcros::ifPredicado}
\begin{multicols}{2}
\begin{tabular}{ll}
	& $Z \leftarrow p(\xDots{V}{n})$ \\
	& $\sif{Z}{L}$
\end{tabular}

\columnbreak
Usando la macro definida en \ref{appendix::otrasMAcros::asignacionFunc}, asignamos a una variable $Z$ el resultado de ejecutar el predicado $p$ con $\xDots{V}{n}$ como variables de entrada. Luego ejecutamos la instrucción IF.
\end{multicols}

%\subsection{Igualdad}\label{appendix::primitivas::igualdad}
%$x = y$
%\begin{demo}1
%	$$igual(x,y) = \alpha(x \resta y)$$
%\end{demo}
%
%\subsection{Divisor}
%$y|x$ si y solo $y$ divide a $x$
%\begin{demo}
%	$$y|x \iff (\exists t)_{\leq x} y\cdot t = x $$
%\end{demo}

\section{Otras definiciones}
\subsection{La función de Ackerman}
Es una función que no es primitiva recursiva.
\begin{align*}
	A(x,y,z) = \left\{
	\begin{tabular}{ll}
		$y + z$ & si $x=0$ \\
		$0$ & si $x = 1$ y $z = 0$ \\
		$1$ & si $x = 2$ y $z = 0$ \\
		$y$ & si $x > 2$ y $z = 0$ \\
		$A(x-1, y, A(x,y,z-1))$ & si $x,z > 0$
	\end{tabular}
	\right.
\end{align*}

\subsubsection{Versión de Robinson $\&$ Peter}
\begin{align*}
B(m,n) = \left\{
\begin{tabular}{ll}
$n + 1$ & si $m=0$ \\
$B(m-1,1)$ & si $m > 0$ y $n = 0$ \\
$B(m-1,B(m,n-1))$ & si $m > 0$ y $n > 0$ \\
\end{tabular}
\right.
\end{align*}
