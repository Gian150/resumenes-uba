	\section{Funciones primitivas recursivas y clases PRC}
	\subsection{Funciones iniciales}
	\paragraph{Función calculable de manera efectiva:} Son funciones que podemos escribir combinando, de alguna manera, funciones más simples que sabemos que ya sabemos que son efectivas.
	
	La idea es que, dado un \textbf{conjunto inicial} de funciones, podamos combinarlas de ciertas formas para conseguir nuevas funciones que permitan realizar cálculos más complejos.
	
	\paragraph{Funciones iniciales:}
	\begin{itemize}
		\item $s(x) = x + 1$
		\item $n(x) = 0$
		\item \textbf{Proyecciones:} $u_i^n(x_1,\dots,x_n) = x_i$ para $i\in\{1,\dots,n\}$
	\end{itemize}

	\paragraph{Composición:} Sea $f:\nat^k\to\nat$ y $\xDots{g}{k}:\nat^n\to\nat$, entonces $h:\nat^n\to\nat$ se obtiene a partir de composición de $f$ y $\xDots{g}{k}$ por composición si:
	
	$$h(\xDots{x}{n}) = \comp{f}{k}{g}{n}$$
	
	En este contexto, una constante $k$ puede ser definida como:
	\begin{align*}
	h(t) &= s^{(k)}(n(t)) \\
	\end{align*}
	
	\paragraph{Recursión primitiva:} $h:\nat^{n+1}\to\nat$ se obtiene a partir de $g\nat^{n+2}\to\nat$ y $f:\nat^n\to\nat$ por recursión primitiva si:
	\begin{align*}
		h(\xDots{x}{n}, 0) &= f(\xDots{x}{n}) \\
		h(\xDots{x}{n}, t+1) &= g(h(\xDots{x}{n}, t), \xDots{x}{n}, t) \\
	\end{align*}
	
	\subsection{Clases Primitivas Recursivas Cerradas (PRC)} 
	
	\paragraph{Función primitiva recursiva:} Función que se puede obtener a partir de las funciones iniciales por un número finito de aplicaciones de composición y recursión primitiva.
	
	\paragraph{Clase PRC:}	Una clase $\mathcal{C}$ de funciones totales es \textbf{PRC} si
	\begin{itemize}
		\item Las funciones iniciales están en $\mathcal{C}$.
		\item Si una función $f$ se obtiene a partir de otras pertenecientes a $\mathcal{C}$ por medio de composición o recursión primitiva, entonces $f$ también está en $\mathcal{C}$.
	\end{itemize}

	\begin{corolario}
La clase de funciones primitivas recursivas es una clase \textit{PRC}.
	\end{corolario}
	
	\begin{teorema}
La clase de funciones totales Turing computables es una clase \textit{PRC}.
	\end{teorema}

	\begin{teorema}\label{teorema::funcPrSiEnTodaClasePrc}
Una función es p.r. si y solo si pertenece a toda clase \textit{PRC}.
	\end{teorema}
	
	\begin{demo}[teorema::funcPrSiEnTodaClasePrc]
		\begin{itemize}
			\item[$\Leftarrow$)] Si una función $f$ pertenece a todas las clases PRC, en particular, pertenece a la clase de funciones primitiva recursivas. Por lo tanto $f$ es primitiva recursiva.
			\item[$\Rightarrow$)] Sea $f$ una función primitiva recursiva y sea $\mathcal{C}$ una clase PRC. Como $f$ es p.r, hay una lista $\xDots{f}{n}$
			tal que
			\begin{itemize}
				\item $f = f_n$ y
				\item $f_i$:
				\begin{itemize}
				\item es una función incial (por lo tanto es p.r y está en $\mathcal{C}$) ó
				\item o se obtiene por composición o recursión primitiva a partir de funciones $f_j$ con $j < i$ (luego tambien está en $\mathcal{C}$)
				\end{itemize}
			\end{itemize}
			Osea que $f=f_n$ o es una función inicial (y por lo tanto está en $\mathcal{C}$) ó se obtiene por composición o recursión primitiva de algunas de las anteriores (que están en $\mathcal{C}$ y por lo tanto $f$ también lo está). Luego $f\in\mathcal{C}$
		\end{itemize}\qed
	\end{demo}

	\begin{corolario}
La clase de funciones primitivas recursivas es la clase PRC más chica.
	\end{corolario}
	
		\begin{corolario}\label{funcPrEsTotYComp}
Toda función p.r. es total y Turing computable
		\end{corolario}
		
	\begin{demo}[funcPrEsTotYComp]
	Sabemos que la clase de funciones totales Turing computables es PRC. Por el teorema, anterior, si $f$ es p.r, entonces $f$ pertence a todas las clases PRC, en particular a la clase de funciones totales Turing computables. \qed
	\end{demo}

\subsection{Funciones primitivas recursivas básicas}
\begin{itemize}
	\item Todas las constantes $k$ están en todas las clases PRC. 
		\begin{itemize}
			\item[] $k = k(t) = s^{(k)}(n(t))$
		\end{itemize}
	\item $suma(x,y) = x + y$
	
	\begin{tabular}{lll}
	$suma(x,0)$ & $=$ & $u_1^1(x)$ \\
	$suma(x, y+1)$ & $=$ & $g(suma(x,y),x,y)$ 
	\\ &  &donde  $g(x_1, x_2,x_3) = s(u_1^3(x_1,x_2,x_3))$
	\end{tabular}
\setlength\columnsep{50pt}
	\begin{multicols}{2}	
			\item $x\cdot y$
			
			\begin{tabular}{lll}
				$x \cdot 0$ & $=$ &$0 = n(x)$\\
				$x \cdot (y+1)$ &$=$  & $g(x \cdot y,x,y) = x + (x\cdot y)$
			\end{tabular}
		
			\item $x^y$
			
			\begin{tabular}{lll}
				$x ^ 0$ &$=$& $1$\\
				$x ^{y+1}$ &$=$ & $g(x^y,x,y)$ = $x \cdot x^y$
			\end{tabular}
			
			\item  $factorial(x) = x!$
			
			\begin{tabular}{lll}
				$0!$ & $=$ & $1 = s(n(x))$\\
				$(x + 1)!$ & $=$ & $g(x!,x) = (x+1) \cdot x!$
			\end{tabular}
		
			\item $x\resta y = \left\{\begin{tabular}{ll}
			$x-y$ & si $x\geq y$ \\
			0 & si no
			\end{tabular}\right.$
			
						\begin{tabular}{lll}
				$x \resta 0$ & $=$ & $x$\\
				$x\resta (y+1)$ & $=$ & $g(x\resta y,x,y) = (x\resta y) \resta 1$
			\end{tabular}
			donde $x\resta 1$ es:
			
									\begin{tabular}{lll}
				$0 \resta 1$ & $=$ & $0$\\
				$(x+1)\resta 1$ & $=$ & $g(x\resta 1,x) = u_2^2(x\resta 1,x) = x$
			\end{tabular}
			\item $\alpha(x) = \left\{\begin{tabular}{ll}
			1 & si $x=0$ \\
			0 & si no
			\end{tabular}\right.$
		
		\end{multicols}
\setlength\columnsep{10pt}
\end{itemize}

	\subsubsection{Predicados primitivos recursivos}
	Los \textbf{predicados} son simplemente funciones que toman valores en $\{0,1\}$.
	\begin{itemize}
	\item 1 se interpreta como verdadero,
	\item 0 se interpreta como falso
	\end{itemize}
	
	Los predicados primitivos recursivos son aquellos representados por funciones primitivas recursivas en $\{0,1\}$.
	

	\subsubsection{Operadores lógicos}
	\begin{teorema}\label{teorema::predPRC}
	Sea $\mathcal{C}$ una clase PRC. Si $p$ y $q$ son predicados en $\mathcal{C}$ entonces $\lnot p$, $p\land q$ y $p\lor q$ están en $\mathcal{C}$
	\end{teorema}

	\begin{demo}[teorema::predPRC]
	\begin{itemize}
	\item $\lnot p = \alpha(p)$
	\item $p\land q = p \cdot q$
	\item $p \lor q = \lnot(\lnot p \land \lnot q)$ 
	\end{itemize}\qed
	\end{demo}
	
	\begin{corolario}
	Si $p$ y $q$ son predicados primitivos recursivos también lo son los predicados $\lnot p$, $p\lor q$ y $p\land q$.
	\end{corolario}
	
	\begin{corolario}
Si $p$ y $q$ son predicados totales Turing computables también lo son los predicados $\lnot p$, $p\lor q$ y $p\land q$
		\end{corolario}
	
	\subsubsection{Definición por casos}
	\begin{teorema}\label{teorema::funcPartEsPRC}
Sea $\mathcal{C}$ una clase PRC. Sean $\xDots{g}{m},h:\nat^n\to\nat$ funciones en $\mathcal{C}$ y sean $\xDots{p}{m}:\nat^n\to\{0,1\}$ predicados mutuamente excluyentes en $\mathcal{C}$. La función:
	
	$$f(\xDots{x}{n}) = \left\{\begin{tabular}{ll}
		$g_1(\xDots{x}{n})$ & si $p_1(\xDots{x}{n})$ \\
	\vdots & \\
$g_m(\xDots{x}{n})$ & si $p_m(\xDots{x}{n})$ \\
	$h(\xDots{x}{n})$ & si no \\
	\end{tabular}
	\right.$$

está en $\mathcal{C}$

	\end{teorema}
\begin{demo}[teorema::funcPartEsPRC]
	\vspace*{-0.5cm}
\begin{align*}
	f(\xDots{x}{n}) = &g_1(\xDots{x}{n})\cdot p_1(\xDots{x}{n}) + \dots \\ &+ g_m(\xDots{x}{n})\cdot p_m(\xDots{x}{n}) \\ &+ h(\xDots{x}{n})\cdot(\lnot p_1(\xDots{x}{n})\land\cdots\land \lnot p_m(\xDots{x}{n}))
\end{align*}

Si algún $p_i$ es verdadero, valdrá $1$ y el resto $0$ porque todos los predicados son mutuamente excluyentes. Al evaluar $f$, obtendremos $$f(\xDots{x}{n}) = g_i(\xDots{x}{n})*p_i(\xDots{x}{n}) = g_i(\xDots{x}{n})\cdot 1 = g_i(\xDots{x}{n})$$

Si todos los predicados son falsos, entonces todas las $g_i$ se multiplican por cero y, además vale la condición \textit{si no} que es la conjuncion de sus negaciones. Y $h$ es multiplicado por 1. \qed
\end{demo}

\subsubsection{Sumatorias, productorias}\label{sec::primitivas::sumatoria}

\begin{teorema}\label{teorema::sumProdSonPRC}
Sea $\mathcal{C}$ una clase PRC. Si $f:\nat^{n+1}\to\nat$ está en $\mathcal{C}$ entonces también están las funciones:

\begin{align*}
	g(y,\xDots{x}{n}) &= \sum_{t=0}^{y} f(t,\xDots{x}{n})) \\
	h(y,\xDots{x}{n}) &= \prod_{t=0}^{y} f(t,\xDots{x}{n})) \\
\end{align*}

\end{teorema}

\begin{demo}[teorema::sumProdSonPRC]
	\vspace*{-0.5cm}
\begin{align*}
	g(0,\xDots{x}{n}) &= f(0, \xDots{x}{n}) \\
	g(t+1,\xDots{x}{n}) &= g(t, \xDots{x}{n}) + f(t+1, \xDots{x}{n})
\end{align*}
Para $h$ hay que hacer lo mismo pero multiplicando en vez de sumar. \qed
\end{demo}

\begin{teorema}
Sea $\mathcal{C}$ una clase PRC. Si $f:\nat^{n+1}\to\nat$ está en $\mathcal{C}$ entonces también están las funciones:
\begin{align*}
	g(y,\xDots{x}{n}) &= \sum_{t=1}^{y} f(t,\xDots{x}{n})) \\
	h(y,\xDots{x}{n}) &= \prod_{t=1}^{y} f(t,\xDots{x}{n})) \\
\end{align*}
\end{teorema}

La demostración es la misma que la anterios, pero los resultados de los casos bases valen $0$ y $1$, respectivamente.

\subsubsection{Cuantificadores acotados}
Sean $p:\nat^{n+1}\to\{0,1\}$ un predicado:
\begin{itemize}
	\item[] $(\forall t)_{\leq y} p(t,\xDots{x}{n})$ es verdadero si y solo si:
	\begin{itemize}
		\item $p(0, \xDots{x}{n})$ es verdadero \textbf{y}
		
		\hspace*{1cm}\vdots
		
		\item $p(y, \xDots{x}{n})$ es verdadero 
	\end{itemize}
\item[] $(\exists t)_{\leq y} p(t,\xDots{x}{n})$ es verdadero si y solo si:
\begin{itemize}
	\item $p(0, \xDots{x}{n})$ es verdadero \textbf{o}
	
	\hspace*{1cm}\vdots
	
	\item $p(y, \xDots{x}{n})$ es verdadero 
\end{itemize}
De la misma manera se pueden definir el existencial y el para todo con $< y$ en vez de $\leq y$.
\end{itemize}

\begin{teorema}\label{teorema::cuantificadoresEstanEnPRC}
 Sean $p:\nat^{n+1}\to\{0,1\}$ un predicado perteneciente a una clase PRC $\mathcal{C}$. Entonces los siguientes predicados también están en $\mathcal{C}$:

\begin{itemize}
	\item[] $(\forall t)_{\leq y} p(t,\xDots{x}{n})$ 
	\item[] $(\exists t)_{\leq y} p(t,\xDots{x}{n})$
\end{itemize}
\end{teorema}

\begin{demo}[teorema::cuantificadoresEstanEnPRC]
\begin{itemize}
	\item[] $(\forall t)_{\leq y} p(t,\xDots{x}{n})$ si y solo si $\prod_{t=1}^{y} f(t,\xDots{x}{n})) = 1$
	\item[] $(\exists t)_{\leq y} p(t,\xDots{x}{n})$ si y solo si $\sum_{t=1}^{y} f(t,\xDots{x}{n})) \neq 0$
\end{itemize}

\vspace*{5mm}
Definimos los cuantificadores en función de la sumatoria, productoria y la comparación de igualdad que están en $\mathcal{C}$ (Ver Seccion \ref{sec::primitivas::sumatoria}). Entonces el para todo y el existencial acotados por $\leq$ pertenece a $\mathcal{C}$. \qed
\end{demo}

\begin{teorema}\label{teorema::cuantificadoresMenoresEstanEnPRC}
Sean $p:\nat^{n+1}\to\{0,1\}$ un predicado perteneciente a una clase PRC $\mathcal{C}$. Entonces los siguientes predicados también están en $\mathcal{C}$:
\begin{itemize}
	\item[] $(\forall t)_{< y} p(t,\xDots{x}{n})$ 
	\item[] $(\exists t)_{< y} p(t,\xDots{x}{n})$
\end{itemize}

\end{teorema}

\begin{demo}[teorema::cuantificadoresMenoresEstanEnPRC]
\begin{itemize}
	\item[] $(\forall t)_{< y} p(t,\xDots{x}{n})$ si y solo si $(\forall t)_{\leq y} (t=y \lor p(t,\xDots{x}{n})$
	\item[] $(\exists t)_{< y} p(t,\xDots{x}{n})$ si y solo si $(\forall t)_{\leq y} (t\neq y \land p(t,\xDots{x}{n})$
\end{itemize}\qed
\end{demo}

\subsubsection{Minimización acotada}
$$
\min\limits_{t\leq y} p(t,\xDots{x}{n})= \left\{
\begin{tabular}{ll}
mínimo $t\leq y$ tal que $p(t,\xDots{x}{n})$ es verdadero & si existe tal $t$ \\
0 & si no
\end{tabular}
\right.
$$

\begin{teorema}\label{teorema::minEstaEnPRC}
Sean $p:\nat^{n+1}\to\{0,1\}$ un predicado de una clase PRC $\mathcal{C}$. Entonces la función $\min\limits_{t\leq y} p(t,\xDots{x}{n})$ también están en $\mathcal{C}$
\end{teorema}

\begin{demo}[teorema::minEstaEnPRC]
 $$\min\limits_{t\leq y} p(t,\xDots{x}{n}) = \sum_{u=0}^{y}\prod_{t=0}^{u}\alpha(p(t,\xDots{x}{n}))$$
 
Veamos que esto funciona. Supongamos que $t_0$ es el mínimo valor que cumple $p$, entonces:
\begin{enumerate}[(1)]
	\item\label{paso::t0} $p(t_0,\xDots{x}{n}) = 1$
	\item\label{paso::t<t0} $p(t,\xDots{x}{n}) = 0$ para todo $t < t_0$
\end{enumerate}

Podemos dividir la sumotaria en dos partes:

$$\underbrace{\sum_{u=0}^{t_0-1}\prod_{t=0}^{u}\alpha(p(t,\xDots{x}{n}))}_{\ref{primero}} + \underbrace{\sum_{u=t_0}^{y}\prod_{t=0}^{u}\alpha(p(t,\xDots{x}{n}))}_{\ref{segundo}}$$
\begin{enumerate}[(1),resume]
	\item\label{primero} Son los primeros $t_0$ términos que corresponden a $u \in [0,t_0-1]$. Por \ref{paso::t<t0} sabemos que $p(t,\xDots{x}{n}) = 0$ para todo $0 \leq t \leq u < t_0$. Entonces:
	$$\sum_{u=0}^{t_0-1}\prod_{t=0}^{u}\alpha(p(t,\xDots{x}{n})) = 
	\sum_{u=0}^{t_0-1}\prod_{t=0}^{u}\alpha(0) = 
	\sum_{u=0}^{t_0-1}\prod_{t=0}^{u}1 = 
	\sum_{u=0}^{t_0-1}1 =
	t_0$$
	\item\label{segundo}Resta ver que todos los términos que corresponden a $u \in [t_0, y]$ dan $0$ (si diese algo más grande, el resultado final superaría a $t_0$).
\end{enumerate}
\end{demo}
\begin{demoPart}[teorema::minEstaEnPRC]
\begin{enumerate}[resume]
	\item[]	Para esto solo hay que notar que todas la productorias restantes van desde $0 \leq t \leq u$ y que $u \geq t_0$, por lo que en todas se encuentra el valor $\alpha(p(t_0,\xDots{x}{n})) = 0$ ($t_0$ cumple $p$). Luego cada una de ellas da cero .
\end{enumerate}  \qed
\end{demoPart}


\subsubsection{Codificación de tuplas}\label{secc::codTuplas}
Definimos la función primitiva recursiva

$$\langle x,y \rangle = 2^x(2\cdot y + 1) - 1$$

\begin{proposicion}
Hay una única solución $(x,y)$ a la ecuación $\langle x,y \rangle = z$.
\begin{itemize}
	\item $x$ es el máximo número tal que $2^x | (z+1)$
	\item $y = (\frac{z+1}{2^x} - 1)/2$
\end{itemize}

\end{proposicion} 
\paragraph{Observadores:} Sea $z = \langle x,y \rangle$:
\begin{itemize}
	\item $l(z) = x$
	\item $r(z) = y$
\end{itemize}

\begin{proposicion}\label{proposicion::ObservadoresParesSonPR}
Los observadores de pares son primitivas recursivas
\end{proposicion}
\begin{demo}[proposicion::ObservadoresParesSonPR]
	Como $x$ e $y$ son menores a $z +1$ tenemos que:
	\begin{itemize}
		\item $l(z) = \min\limits_{x\leq z} \left(\left(\exists y\right)_{\leq z}~ z = \left\langle x,y\right\rangle\right)$
		\item $r(z) = \min\limits_{y\leq z} \left(\left(\exists x\right)_{\leq z}~ z = \left\langle x,y\right\rangle\right)$
	\end{itemize}
\end{demo}

\subsubsection{Codificación de secuencias}\label{secc::codSecuencias}
El \textbf{número de Gödel} de la secuencia $\xDots{a}{n}$ es el número:
$$[\xDots{a}{n}] = \prod_{i=1}^{n} p_i^{a_i}$$
donde $p_i$ es el $i-$ésimo primo ($i\geq 1$)

\begin{teorema}
Si $\xDots{a}{n} = \xDots{b}{n}$ entonces $a_i = b_i$ para todo $i\in\{1,\dots,n\}$.
\end{teorema}

\paragraph{Observación:} La demostración del teorema se basa en que la factorización en primos de cualquier número es única.

\paragraph{Observación:} $[\xDots{a}{n}] = [\xDots{a}{n},0] = [\xDots{a}{n},0,0]= \dots$

\paragraph{Observadores:} Sea $x=[\xDots{a}{n}]$:
\begin{itemize}
	\item $x[i] = a_i$
	\item $|x| =$ longitud de $x$
\end{itemize}

\begin{proposicion}\label{proposicion::observadoresSecuenciaSonPR}
Los observadores de secuencias son primitivas recursivas
\end{proposicion}

\begin{demo}[proposicion::observadoresSecuenciaSonPR]
	\begin{itemize}
		\item $x[i] = \min\limits_{t\leq x}\left(\lnot\left(p_i^{t+1}|x\right)\right)$
			\begin{itemize}
				\item [] Es el mínimo $t$ tal que $p_i^{t+1}$ no divide a $x$. La función $p_i$ que nos devuelve el $i$-ésimo primo es primitiva recursiva y el predicado de divisibilidad son primitivos recursivos (Ver apéndice \ref{appendix::funcpr})
			\end{itemize}
		\item $|x| = \min\limits_{i\leq x}\left(x[i]\neq 0 \land \left(\forall j\right)_{\leq x} \left(j\leq i \lor x[j] = 0\right)\right)$
	\end{itemize}
\end{demo}
